\documentclass{beamer}
\mode<presentation>
{
\usetheme{Warsaw}
% \setbeamercovered{transparent}
}
\usepackage{amsmath,amsfonts,array,eepic,graphics}
%\newcolumntype{d}{D{.}{.}{-1}}
\usepackage{times}
%\usepackage[T1]{fontenc}
\usepackage{subfigure}
\title[RMRC 2018 Solution Sketches]
{2018 Rocky Mountain Regional \\ Programming Contest \\ \ \\ Solution Sketches}
%\author % (optional, use only with lots of authors)
%{Howard Cheng}
% - Use the \inst command only if there are several affiliations.
% - Keep it simple, no one is interested in your street address.
\date
{}
% Delete this, if you do not want the table of contents to pop up at
% the beginning of each subsection:
%\AtBeginSubsection[]
%{
% \begin{frame}<beamer>
% \frametitle{Outline}
% \tableofcontents[currentsection,currentsubsection]
% \end{frame}
%}
% If you wish to uncover everything in a step-wise fashion, uncomment
% the following command:
%\beamerdefaultoverlayspecification{<+->}



\begin{document}
\begin{frame}
\titlepage
\end{frame}
\begin{frame}
\frametitle{Credits}
\begin{itemize}
\setlength\itemsep{0.5\baselineskip}
\item Howard Cheng
\item Brandon Fuller
\item Darcy Best
\item Zachary Friggstad
\item Christopher Painter-Wakefield
\item Darko Aleksic
\item Warren MacEvoy
\item Greg Hamerly
\end{itemize}
\end{frame}


\begin{frame}
\frametitle{A - Quality-Adjusted Life-Year (63/72)}
\begin{itemize}
\setlength\itemsep{0.5\baselineskip}
\item Simple, do what is asked of you.
\end{itemize}
\end{frame}


\begin{frame}
\frametitle{D - H-Index (36/246)}
\begin{itemize}
  \setlength\itemsep{0.5\baselineskip}
\item Find the maximum $h$ such that there are at least $h$ values that are at least $h$.
\item Sort the array (largest to smallest)
\item For each $h$ (from $0$ to $n$), the largest $h$ numbers are at the beginning of the array, so just check if the $h$-th value is at least $h$.
\end{itemize}
\end{frame}


\begin{frame}
\frametitle{G - Neighborhood Watch (36/141)}
\begin{itemize}
\setlength\itemsep{0.5\baselineskip}
\item How many intervals contain a ``special'' house?
\item Count the number of bad paths and subtract from total number of intervals.
\item Look at the size of the gaps between special houses.
\item If the gap has $g$ houses, then the number of bad paths in that gap is $g(g+1)/2$.
\end{itemize}
\end{frame}


\begin{frame}
\frametitle{H - Small Schedule (35/114)}
\begin{itemize}
\setlength\itemsep{0.5\baselineskip}
\item Bin packing with two sizes
\item Fill with large ones, then with $1$'s
\item Be careful of off-by-one errors
\end{itemize}
\end{frame}

\begin{frame}
\frametitle{C - Forest for the Trees (14/86)}
\begin{itemize}
\setlength\itemsep{0.5\baselineskip}
\item Which trees can obstruct the view?
\item $(k*x_b/g,k*y_b/g)$ where $g=gcd(x_b,y_b)$ is the next tree in the way.
\item Look at the values of $k$ such that the corresponding trees are in the rectangle.
\item Can be done in constant time by looking at linear inequalities.
\end{itemize}
\end{frame}


\begin{frame}
\frametitle{E - Driving Lanes (5/9)}
\begin{itemize}
\setlength\itemsep{0.5\baselineskip}
\item Dynamic programming
\item Define $f(n,L) = $ the shortest distance to drive the first $n$ straightaways and be in lane $L$.
\item For each state, try changing to each of the other lanes (if possible).
\end{itemize}
\end{frame}

\begin{frame}
\frametitle{I - Mr. Plow King (5/14)}
\begin{itemize}
\setlength\itemsep{0.3\baselineskip}
\item How to maximize the minimum spanning tree?
\item Kruskal's algorithm tells us that we should greedily take the shortest edge if it doesn't create a cycle.
\item You \textit{must} include edges 1 and 2. The only way to not include edge 3 is to make a triangle (with 1, 2, 3).
\item Then you \textit{must} include edge 4. To not use edge 5 and 6, connect those edges into your connected component.
\item Then you \textit{must} include edge 7. Etc.
\item At some point, you can use all remaining edges in the MST. Put them in a path.
\item In general:
  \begin{itemize}
   \item Make a giant connected component with lots of small edges.
   \item Make a long path of large edges.
  \end{itemize}
\end{itemize}
\end{frame}


\begin{frame}
\frametitle{J - Rainbow Road Race (3/3)}
\begin{itemize}
\setlength\itemsep{0.5\baselineskip}
\item Travel through edges to get colors. Return to the start location with all colors with the shortest travel time.
\item Create a new graph where the nodes are the pair: (location, set of colors you have).
\item For each edge (from location $i$ to $j$) create an edge from $(i, S)$ to $(j, S \cup \{$color of edge$ \})$ for all possible color sets $S$.
\item Run a shortest path algorithm from $(1, \{ \})$ to $(1, \{ R,O,Y,G,B,I,V \} )$.
\end{itemize}
\end{frame}

\begin{frame}
\frametitle{B - Gwen's Gift (2/10)}
\begin{itemize}
\setlength\itemsep{0.5\baselineskip}
\item There are $(n-1)!$ valid sequences, for each first element there are $(n-2)!$ valid sequences and so on
\item Keep track of the sums modulo $n$.
\item For each position, go from $1$ to $n-1$ skipping the \textit{invalid} elements until you get to the desired index
\item An element is invalid if (a) the prefix sum is $0$ or (b) we have already seen the prefix sum (both mod $n$)
\item $20! > 10^{18}$ - we care only about the last $20$ elements, the rest are all $1$'s
\end{itemize}
\end{frame}


\begin{frame}
\frametitle{B - Gwen's Gift (2/10) (...continued...)}
\begin{itemize}
\setlength\itemsep{0.5\baselineskip}
\item Another way to think about the problem: the partial sums of the
  array (modulo $n$) are non-zero and distinct.
\item So the partial sums modulo $N$ form a permutation of $1, 2, \ldots, n-1$.
\item Generate the $k$th permutation (but order based on the original sequence).
\end{itemize}
\end{frame}

\begin{frame}
\frametitle{F - Treasure Spotting (0/23)}
\begin{itemize}
\setlength\itemsep{0.5\baselineskip}
\item Relatively simple geometry, most cases covered in samples
\item Line segment intersection
\item Use integer arithmetic (do \textbf{not} use floating point)
\end{itemize}
\end{frame}

\end{document}
